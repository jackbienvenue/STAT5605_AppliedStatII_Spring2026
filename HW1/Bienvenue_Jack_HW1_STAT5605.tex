% Options for packages loaded elsewhere
\PassOptionsToPackage{unicode}{hyperref}
\PassOptionsToPackage{hyphens}{url}
\documentclass[
  12pt,
]{article}
\usepackage{xcolor}
\usepackage[margin=1in]{geometry}
\usepackage{amsmath,amssymb}
\setcounter{secnumdepth}{5}
\usepackage{iftex}
\ifPDFTeX
  \usepackage[T1]{fontenc}
  \usepackage[utf8]{inputenc}
  \usepackage{textcomp} % provide euro and other symbols
\else % if luatex or xetex
  \usepackage{unicode-math} % this also loads fontspec
  \defaultfontfeatures{Scale=MatchLowercase}
  \defaultfontfeatures[\rmfamily]{Ligatures=TeX,Scale=1}
\fi
\usepackage{lmodern}
\ifPDFTeX\else
  % xetex/luatex font selection
\fi
% Use upquote if available, for straight quotes in verbatim environments
\IfFileExists{upquote.sty}{\usepackage{upquote}}{}
\IfFileExists{microtype.sty}{% use microtype if available
  \usepackage[]{microtype}
  \UseMicrotypeSet[protrusion]{basicmath} % disable protrusion for tt fonts
}{}
\makeatletter
\@ifundefined{KOMAClassName}{% if non-KOMA class
  \IfFileExists{parskip.sty}{%
    \usepackage{parskip}
  }{% else
    \setlength{\parindent}{0pt}
    \setlength{\parskip}{6pt plus 2pt minus 1pt}}
}{% if KOMA class
  \KOMAoptions{parskip=half}}
\makeatother
\usepackage{color}
\usepackage{fancyvrb}
\newcommand{\VerbBar}{|}
\newcommand{\VERB}{\Verb[commandchars=\\\{\}]}
\DefineVerbatimEnvironment{Highlighting}{Verbatim}{commandchars=\\\{\}}
% Add ',fontsize=\small' for more characters per line
\usepackage{framed}
\definecolor{shadecolor}{RGB}{248,248,248}
\newenvironment{Shaded}{\begin{snugshade}}{\end{snugshade}}
\newcommand{\AlertTok}[1]{\textcolor[rgb]{0.94,0.16,0.16}{#1}}
\newcommand{\AnnotationTok}[1]{\textcolor[rgb]{0.56,0.35,0.01}{\textbf{\textit{#1}}}}
\newcommand{\AttributeTok}[1]{\textcolor[rgb]{0.13,0.29,0.53}{#1}}
\newcommand{\BaseNTok}[1]{\textcolor[rgb]{0.00,0.00,0.81}{#1}}
\newcommand{\BuiltInTok}[1]{#1}
\newcommand{\CharTok}[1]{\textcolor[rgb]{0.31,0.60,0.02}{#1}}
\newcommand{\CommentTok}[1]{\textcolor[rgb]{0.56,0.35,0.01}{\textit{#1}}}
\newcommand{\CommentVarTok}[1]{\textcolor[rgb]{0.56,0.35,0.01}{\textbf{\textit{#1}}}}
\newcommand{\ConstantTok}[1]{\textcolor[rgb]{0.56,0.35,0.01}{#1}}
\newcommand{\ControlFlowTok}[1]{\textcolor[rgb]{0.13,0.29,0.53}{\textbf{#1}}}
\newcommand{\DataTypeTok}[1]{\textcolor[rgb]{0.13,0.29,0.53}{#1}}
\newcommand{\DecValTok}[1]{\textcolor[rgb]{0.00,0.00,0.81}{#1}}
\newcommand{\DocumentationTok}[1]{\textcolor[rgb]{0.56,0.35,0.01}{\textbf{\textit{#1}}}}
\newcommand{\ErrorTok}[1]{\textcolor[rgb]{0.64,0.00,0.00}{\textbf{#1}}}
\newcommand{\ExtensionTok}[1]{#1}
\newcommand{\FloatTok}[1]{\textcolor[rgb]{0.00,0.00,0.81}{#1}}
\newcommand{\FunctionTok}[1]{\textcolor[rgb]{0.13,0.29,0.53}{\textbf{#1}}}
\newcommand{\ImportTok}[1]{#1}
\newcommand{\InformationTok}[1]{\textcolor[rgb]{0.56,0.35,0.01}{\textbf{\textit{#1}}}}
\newcommand{\KeywordTok}[1]{\textcolor[rgb]{0.13,0.29,0.53}{\textbf{#1}}}
\newcommand{\NormalTok}[1]{#1}
\newcommand{\OperatorTok}[1]{\textcolor[rgb]{0.81,0.36,0.00}{\textbf{#1}}}
\newcommand{\OtherTok}[1]{\textcolor[rgb]{0.56,0.35,0.01}{#1}}
\newcommand{\PreprocessorTok}[1]{\textcolor[rgb]{0.56,0.35,0.01}{\textit{#1}}}
\newcommand{\RegionMarkerTok}[1]{#1}
\newcommand{\SpecialCharTok}[1]{\textcolor[rgb]{0.81,0.36,0.00}{\textbf{#1}}}
\newcommand{\SpecialStringTok}[1]{\textcolor[rgb]{0.31,0.60,0.02}{#1}}
\newcommand{\StringTok}[1]{\textcolor[rgb]{0.31,0.60,0.02}{#1}}
\newcommand{\VariableTok}[1]{\textcolor[rgb]{0.00,0.00,0.00}{#1}}
\newcommand{\VerbatimStringTok}[1]{\textcolor[rgb]{0.31,0.60,0.02}{#1}}
\newcommand{\WarningTok}[1]{\textcolor[rgb]{0.56,0.35,0.01}{\textbf{\textit{#1}}}}
\usepackage{longtable,booktabs,array}
\usepackage{calc} % for calculating minipage widths
% Correct order of tables after \paragraph or \subparagraph
\usepackage{etoolbox}
\makeatletter
\patchcmd\longtable{\par}{\if@noskipsec\mbox{}\fi\par}{}{}
\makeatother
% Allow footnotes in longtable head/foot
\IfFileExists{footnotehyper.sty}{\usepackage{footnotehyper}}{\usepackage{footnote}}
\makesavenoteenv{longtable}
\usepackage{graphicx}
\makeatletter
\newsavebox\pandoc@box
\newcommand*\pandocbounded[1]{% scales image to fit in text height/width
  \sbox\pandoc@box{#1}%
  \Gscale@div\@tempa{\textheight}{\dimexpr\ht\pandoc@box+\dp\pandoc@box\relax}%
  \Gscale@div\@tempb{\linewidth}{\wd\pandoc@box}%
  \ifdim\@tempb\p@<\@tempa\p@\let\@tempa\@tempb\fi% select the smaller of both
  \ifdim\@tempa\p@<\p@\scalebox{\@tempa}{\usebox\pandoc@box}%
  \else\usebox{\pandoc@box}%
  \fi%
}
% Set default figure placement to htbp
\def\fps@figure{htbp}
\makeatother
\setlength{\emergencystretch}{3em} % prevent overfull lines
\providecommand{\tightlist}{%
  \setlength{\itemsep}{0pt}\setlength{\parskip}{0pt}}
\usepackage[most]{tcolorbox}
\newtcolorbox{bluebox}{
  colback=blue!5!white,    % Light blue background
  colframe=blue!80!black,  % Darker blue border
  boxrule=0.8pt,
  arc=4pt,
  left=6pt,
  right=6pt,
  top=6pt,
  bottom=6pt
}
\usepackage{amsmath}
\usepackage{amssymb}
\usepackage{titling}
\setlength{\droptitle}{1em}
\usepackage{bookmark}
\IfFileExists{xurl.sty}{\usepackage{xurl}}{} % add URL line breaks if available
\urlstyle{same}
\hypersetup{
  pdftitle={STAT 5605 Homework 1},
  pdfauthor={Jack Bienvenue},
  hidelinks,
  pdfcreator={LaTeX via pandoc}}

\title{STAT 5605 Homework 1}
\usepackage{etoolbox}
\makeatletter
\providecommand{\subtitle}[1]{% add subtitle to \maketitle
  \apptocmd{\@title}{\par {\large #1 \par}}{}{}
}
\makeatother
\subtitle{February 5, 2026}
\author{Jack Bienvenue}
\date{}

\begin{document}
\maketitle

{
\setcounter{tocdepth}{2}
\tableofcontents
}
\section{Problem 1}\label{problem-1}

When asked to state the simple linear regression model, a student wrote as follows: \(E[Y_i]=\beta_0+\beta_1 X_i + \epsilon_i\). Do you agree?

\begin{bluebox}

I do \textbf{not} agree with the student's statement of the simple linear regression model. While it is very close to being correct, there is a conceptual mistake. In the correct simple linear regression (SLR) model, $Y_i = \beta_0 + \beta_1X_i + \epsilon_i, \quad i=1,  \dots,n$, we have a few components: \newline

\begin{enumerate}
 \item The value of the outcome variable for a specific observation, $Y_i$,
 \item The SLR model intercept, $\beta_0$,
 \item The SLR model slope, $\beta_1$,
 \item The value of the input variable for a specific observation, $X_i$,
 \item and the random error associated with the observation's outcome variable value, $\epsilon_i$.
\end{enumerate}

The student incorrectly added an expectation function around $Y_i$. This is incorrect as the \textit{expectation} of random variable $Y_i$ does not include the random error associated with the realization. The expectation of $Y_i$ given $X_i$ \textbf{under our SLR model} is actually $E[Y_i | X_i] = \beta_0 + \beta_1X_i$.

\end{bluebox}

\section{Problem 2}\label{problem-2}

In a simulation exercise, regression model on page 19 of note 1 applies with \(\beta_0=100\), \(\beta_1=20\), and \(\sigma^2=25\). An observation on \(Y\) will be made for \(X=5\).

\subsection{(a)}\label{a}

Can you state the exact probability that \(Y\) will fall between 195 and 205? Explain.

\begin{bluebox}
Let's begin by building our SLR model using the coefficients given by the problem description: \newline

$$ 
Y = 100 + (20\times5) + \epsilon_i, \text{with assumptions:}
$$
\begin{enumerate}
  \item Expectation of errors is 0, i.e. $E(\epsilon_i) = 0$,
  \item Homoscedasticity, i.e. $Var(\epsilon_i) = \sigma^2$,
  \item Errors are uncorrelated between observations.
\end{enumerate}

In this case, we \emph{cannot} state the exact probability that $Y$ will fall between $195$ and $205$ because \emph{although we have all relevant information for the important parameters} ($\beta_0, \beta_1, Var(\epsilon_i)$) \emph{in our model}, we do not have information about the specific distributional shape of the errors ($\epsilon_i$), disallowing us to make statements about the exact probability of an observation's $Y$ value to fall in a given interval. 

\end{bluebox}

\subsection{(b)}\label{b}

If the normal error is assumed, can you now state the exact probability that \(Y\) will fall between 195 and 205? If so, state it.

\begin{Shaded}
\begin{Highlighting}[]
\CommentTok{\# Calculate the probabilities}
\DocumentationTok{\#\# P(Y\textless{}=195)}
\NormalTok{prob\_195 }\OtherTok{\textless{}{-}} \FunctionTok{pnorm}\NormalTok{(}\DecValTok{195}\NormalTok{, }\AttributeTok{mean =} \DecValTok{200}\NormalTok{, }\AttributeTok{sd =} \DecValTok{5}\NormalTok{)}
\DocumentationTok{\#\# P(Y\textless{}=205)}
\NormalTok{prob\_205 }\OtherTok{\textless{}{-}} \FunctionTok{pnorm}\NormalTok{(}\DecValTok{205}\NormalTok{, }\AttributeTok{mean =} \DecValTok{200}\NormalTok{, }\AttributeTok{sd =} \DecValTok{5}\NormalTok{)}
\DocumentationTok{\#\# P(Y\textless{}=205) {-} P(Y\textless{}=195) = P(195 \textless{}= Y \textless{}= 205)}
\NormalTok{prob\_bw\_195\_205 }\OtherTok{\textless{}{-}}\NormalTok{ prob\_205 }\SpecialCharTok{{-}}\NormalTok{ prob\_195}
\FunctionTok{cat}\NormalTok{(}\StringTok{"Probability of Y being in [195, 205]:"}\NormalTok{, prob\_bw\_195\_205)}
\end{Highlighting}
\end{Shaded}

\begin{verbatim}
## Probability of Y being in [195, 205]: 0.6826895
\end{verbatim}

\begin{bluebox}

If we assume that errors are normally distributed, $\epsilon_i \sim N(0, \sigma^2)$, we are now able to calculate probabilities for $Y$ falling in certain intervals. These intervals are ``exact" under a normal error ($N(0, \sigma^2)$) assumption, and will be exact if this assumption accurately describes the error distribution (otherwise, the calculated probability will be an approximation).

To calculate the probability that $Y$ falls between $195$ and $205$ given that $\beta_0 = 100$, $\beta_1 = 20$, and $X = 5$, we should first use this information to obtain $E[Y_{X=5}]$, the ``center" for our errors: $Y = 100 + (20\times5) = 200$. From here, we can use R's \texttt{pnorm()} function, setting the mean to $200$ (our mean-zero errors are centered at $E[Y_{X=5}]$) and standard deviation to $5$ (our \emph{variance} for the errors is $25$, therefore the relevant standard deviation is $\sqrt{25}=5$). We can calculate the lower-tail-to-quantile probabilities and subtract the lower probability (corresponding to $195$) from the higher probability (corresponding to $205$) to find the probability of $Y$ falling in the 195-205 range. Based on the empirical rule, we expect this probability to be around $68\%$ because the bounds are one standard deviation removed from the mean. We do find this result to be true in our calculation, as the computed probability is $0.6826895$. This probability is exact if the normal assumption on the errors we made is true and if all parameter values were the true population parameters. Otherwise, this probability is approximate. 

\end{bluebox}

\section{Problem 3}\label{problem-3}

The regression function relating production output by an employee after taking a training program (\(Y\)) to the production output before the training program (\(X\)) is \(E\{Y\}= 20+0.95X\), where \(X\) ranges from 40 to 100. An observer concludes that the training program does not raise production output on the average because \(\beta_1\) is not greater than 1.0. Comment.

\begin{Shaded}
\begin{Highlighting}[]
\CommentTok{\# Create plot}
\NormalTok{x }\OtherTok{\textless{}{-}} \FunctionTok{seq}\NormalTok{(}\DecValTok{40}\NormalTok{, }\DecValTok{100}\NormalTok{, }\AttributeTok{by =} \DecValTok{1}\NormalTok{)}
\NormalTok{y }\OtherTok{\textless{}{-}} \DecValTok{20} \SpecialCharTok{+} \FloatTok{0.95} \SpecialCharTok{*}\NormalTok{ x}
\NormalTok{df }\OtherTok{\textless{}{-}} \FunctionTok{data.frame}\NormalTok{(}\AttributeTok{x =}\NormalTok{ x, }\AttributeTok{y =}\NormalTok{ y)}
\FunctionTok{ggplot}\NormalTok{(df, }\FunctionTok{aes}\NormalTok{(}\AttributeTok{x =}\NormalTok{ x, }\AttributeTok{y =}\NormalTok{ y)) }\SpecialCharTok{+}
  \FunctionTok{geom\_line}\NormalTok{(}\AttributeTok{linewidth =} \FloatTok{1.2}\NormalTok{, }\AttributeTok{color =} \StringTok{"cornflowerblue"}\NormalTok{) }\SpecialCharTok{+}
  \FunctionTok{labs}\NormalTok{( }\AttributeTok{title =} \StringTok{"SLR: Employee Productivity Pre{-} and Post{-}Training"}\NormalTok{,}
    \AttributeTok{x =} \StringTok{"X {-} Pre{-}Training Productivity"}\NormalTok{,}
    \AttributeTok{y =} \StringTok{"E[Y] {-} Post{-}Training Expected Productivity"}\NormalTok{ ) }\SpecialCharTok{+}
  \FunctionTok{theme\_minimal}\NormalTok{(}\AttributeTok{base\_size =} \DecValTok{13}\NormalTok{) }\SpecialCharTok{+}
  \FunctionTok{theme}\NormalTok{( }\AttributeTok{plot.title =} \FunctionTok{element\_text}\NormalTok{(}\AttributeTok{face =} \StringTok{"bold"}\NormalTok{, }\AttributeTok{hjust =} \FloatTok{0.5}\NormalTok{),}
    \AttributeTok{axis.title =} \FunctionTok{element\_text}\NormalTok{(}\AttributeTok{face =} \StringTok{"bold"}\NormalTok{))}
\end{Highlighting}
\end{Shaded}

\pandocbounded{\includegraphics[keepaspectratio]{Bienvenue_Jack_HW1_STAT5605_files/figure-latex/unnamed-chunk-2-1.pdf}}

\begin{bluebox}

While the observer may be wary of the efficacy of the training program due to the fact that $\beta_1 < 1$, we can assure them that there is a positive effect of the training program on employee productivity on average in this linear model. When we plot our regression line, we can easily observe that employees across all pre-training productivity levels experienced an increase in productivity after undergoing training, on average using this model. Even in such a simple linear model, the regression line is being defined by two different parameters, $\beta_0$ and $\beta_1$, and therefore the effect is being ``split" between these contributors. The \emph{combination} of $\beta_0 = 20$ and $\beta_1 = 0.95$ actually yields a regression that would suggest that, on average, there is a positive productivity effect associated with undergoing training. A coefficient $\beta_1 < 1 \nRightarrow E[Y|X] < X$, because algebraically, $20 + 0.95X > X$ for $X \in [40,100]$.

\end{bluebox}

\section{Problem 4}\label{problem-4}

Evaluate the following statement: ``For the least squares method to be fully valid, it is required that the distribution of \(Y\) be normal.''

\begin{bluebox}

\end{bluebox}

\section{Problem 5}\label{problem-5}

According to page 36 of note 1, \(\sum^n_{i=1} e_i = 0\) when a SLR model is fitted to a set of \(n\) cases by the method of least squares. Is it also true that \(\sum^n_{i=1} \epsilon_i =0\)? Comment.

\begin{bluebox}

\end{bluebox}

\section{Problem 6}\label{problem-6}

The least squares regression line for a given set of data with a sample size of \(n=20\) is \(\hat{Y}=-42 + 0.9 X\) (i.e., \(b_0=-42\) and \(b_1=0.9\)). The MSE of the fitted simple linear regression (SLR) model is 0.14, and the standard error of \(b_1\) (i.e., \(se(b_1)\)) is 0.016. Suppose \(\bar{X}=200\). Answer the following questions and additionally provide references for the pertinent equation numbers from the notes and/or textbook.

\subsection{(a)}\label{a-1}

What is the fitted value of \(Y\) at \(X=220\).

\begin{bluebox}

\end{bluebox}

\subsection{(b)}\label{b-1}

Compute the standard error of \(b_0\).

\begin{bluebox}

\end{bluebox}

\subsection{(c)}\label{c}

Find \(\bar{Y}\).

\begin{bluebox}

\end{bluebox}

\subsection{(d)}\label{d}

What are \(S_{XX}\) and \(S_{XY}\) for this data set?

\begin{bluebox}

\end{bluebox}

\subsection{(e)}\label{e}

Compute \(\mbox{corr}(b_0,b_1)\).

\begin{bluebox}

\end{bluebox}

\section{Problem 7}\label{problem-7}

Suppose you are given \(n\) pairs of observations \((X_1,Y_1), \dots, (X_n,Y_n)\).

\subsection{(a)}\label{a-2}

Describe an empirical Q-Q plot and a scatter plot for this data set?

\begin{bluebox}

\end{bluebox}

\subsection{(b)}\label{b-2}

Can an empirical Q-Q plot be identical to the respective scatter plot for certain data set? If so, when would this happen?

\begin{bluebox}

\end{bluebox}

\subsection{(c)}\label{c-1}

To gain better understanding betweenthese two types of plots, draw the empirical Q-Q plot and the scatter plot for the Ratings of TV Shows Data in Example 2 from the HuskyCT class website. Provide a brief discussion.

\begin{Shaded}
\begin{Highlighting}[]
\NormalTok{ratings }\OtherTok{=} \FunctionTok{read.csv}\NormalTok{(}\StringTok{"ratings.csv"}\NormalTok{)}
\FunctionTok{head}\NormalTok{(ratings)}
\end{Highlighting}
\end{Shaded}

\begin{verbatim}
##     X   Y
## 1 2.5 3.8
## 2 2.7 4.1
## 3 2.9 5.8
## 4 3.1 4.8
## 5 3.3 5.7
## 6 3.5 4.4
\end{verbatim}

\begin{Shaded}
\begin{Highlighting}[]
\FunctionTok{attach}\NormalTok{(ratings)}
\FunctionTok{library}\NormalTok{(ggplot2)}
\FunctionTok{plot}\NormalTok{(X,Y, }\AttributeTok{main=}\StringTok{"Scatterplot of News Ratings vs Lead Ratings"}\NormalTok{,}
     \AttributeTok{ylab=}\StringTok{"News Ratings"}\NormalTok{, }\AttributeTok{xlab=}\StringTok{"Lead Ratings"}\NormalTok{)}
\end{Highlighting}
\end{Shaded}

\pandocbounded{\includegraphics[keepaspectratio]{Bienvenue_Jack_HW1_STAT5605_files/figure-latex/unnamed-chunk-12-1.pdf}}

\begin{Shaded}
\begin{Highlighting}[]
\NormalTok{sx }\OtherTok{\textless{}{-}} \FunctionTok{sort}\NormalTok{(X); sy }\OtherTok{\textless{}{-}} \FunctionTok{sort}\NormalTok{(Y)}
\NormalTok{lenx }\OtherTok{\textless{}{-}} \FunctionTok{length}\NormalTok{(sx); leny }\OtherTok{\textless{}{-}} \FunctionTok{length}\NormalTok{(sy)}
\FunctionTok{ggplot}\NormalTok{() }\SpecialCharTok{+} \FunctionTok{geom\_point}\NormalTok{(}\FunctionTok{aes}\NormalTok{(}\AttributeTok{x =}\NormalTok{ sx, }\AttributeTok{y =}\NormalTok{ sy)) }\SpecialCharTok{+}
\FunctionTok{ggtitle}\NormalTok{(}\StringTok{"Empirical Quantile Plot of X and Y"}\NormalTok{) }\SpecialCharTok{+}
\FunctionTok{xlab}\NormalTok{(}\StringTok{"X"}\NormalTok{) }\SpecialCharTok{+} \FunctionTok{ylab}\NormalTok{(}\StringTok{"Y"}\NormalTok{)}
\end{Highlighting}
\end{Shaded}

\pandocbounded{\includegraphics[keepaspectratio]{Bienvenue_Jack_HW1_STAT5605_files/figure-latex/unnamed-chunk-12-2.pdf}}

\begin{bluebox}

\end{bluebox}

\section{Problem 8}\label{problem-8}

Suppose you are given \(n\) pairs of observations \((X_1,Y_1), \dots, (X_n,Y_n)\). Let \(e_i\) denote the residual for the \(i^{th}\) observation calculated based on the Least Squares method. Using algebra of least squares, argue that weighted sum of residuals, with \(i^{th}\) residual weighted by the corresponding \(Y_i\), is SSE.

\begin{bluebox}

\end{bluebox}

\section{Problem 9}\label{problem-9}

A student was investigating from a large sample whether variables \(Y_1\) and \(Y_2\) follow a bivariate normal distribution. The student obtained the residuals when regressing \(Y_1\) on \(Y_2\), and also obtained the residuals when regressing \(Y_2\) on \(Y_1\), and then prepared a normal probability plot for each set of residuals. Do these two normal probability plots provide sufficient information for determining whether the two variables follow a bivariate normal distribution? Explain.

\begin{bluebox}

\end{bluebox}

\section{Problem 10}\label{problem-10}

The data below show, for a consumer finance company operating in seven cities, the number of competing loan companies operating in a city (\(X_i\)) and the number per thousand of delinquent loans made in that city (\(Y_i\)):

\begin{center}
\begin{tabular}{cccccccc}
$X_i$ & 4 & 1 & 2 & 3 & 3 & 4 & 2\\
$Y_i$ & 18 & 4 & 9 & 14 & 16 & 20 & 8\\
\end{tabular}

\end{center}

For a simple linear regression analysis, let \(X\) denote the design matrix and \(Y\) denote the column vector of responses for the dataset in reference above.
Compute \(X'X,\) \(X'Y,\) \((X'X)^{-1},\) and use these results to find the estimated vector \(\textbf{b}=\begin{pmatrix}
        b_0 \\
        b_1
\end{pmatrix}\) of the regression coefficients.

\begin{Shaded}
\begin{Highlighting}[]
\NormalTok{one}\OtherTok{=}\FunctionTok{rep}\NormalTok{(}\DecValTok{1}\NormalTok{,}\DecValTok{7}\NormalTok{)}
\NormalTok{x1}\OtherTok{=}\FunctionTok{c}\NormalTok{(}\DecValTok{4}\NormalTok{, }\DecValTok{1}\NormalTok{, }\DecValTok{2}\NormalTok{, }\DecValTok{3}\NormalTok{, }\DecValTok{3}\NormalTok{, }\DecValTok{4}\NormalTok{, }\DecValTok{2}\NormalTok{)}
\NormalTok{X}\OtherTok{=}\FunctionTok{t}\NormalTok{(}\FunctionTok{rbind}\NormalTok{(one,x1))}
\NormalTok{X}
\end{Highlighting}
\end{Shaded}

\begin{verbatim}
##      one x1
## [1,]   1  4
## [2,]   1  1
## [3,]   1  2
## [4,]   1  3
## [5,]   1  3
## [6,]   1  4
## [7,]   1  2
\end{verbatim}

\begin{Shaded}
\begin{Highlighting}[]
\NormalTok{Y}\OtherTok{=}\FunctionTok{c}\NormalTok{(}\DecValTok{18}\NormalTok{, }\DecValTok{4}\NormalTok{, }\DecValTok{9}\NormalTok{, }\DecValTok{14}\NormalTok{, }\DecValTok{16}\NormalTok{, }\DecValTok{20}\NormalTok{, }\DecValTok{8}\NormalTok{)}
\NormalTok{Y}
\end{Highlighting}
\end{Shaded}

\begin{verbatim}
## [1] 18  4  9 14 16 20  8
\end{verbatim}

\begin{Shaded}
\begin{Highlighting}[]
\NormalTok{XTY}\OtherTok{=}\FunctionTok{t}\NormalTok{(X) }\SpecialCharTok{\%*\%}\NormalTok{ Y}
\NormalTok{XTY}
\end{Highlighting}
\end{Shaded}

\begin{verbatim}
##     [,1]
## one   89
## x1   280
\end{verbatim}

\begin{bluebox}

\end{bluebox}

\end{document}
